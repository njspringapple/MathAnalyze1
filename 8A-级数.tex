\documentclass[a4paper,12pt]{article}

% 基本设置包
\usepackage[utf8]{inputenc}
\usepackage[T1]{fontenc}
\usepackage{geometry}
\usepackage{setspace}
\usepackage{fancyhdr}
\usepackage{titlesec}
\usepackage{fontspec}
\usepackage{xcolor}
\usepackage{footmisc}
\usepackage{amsmath,amssymb}

% 代码高亮支持
\usepackage{fancyvrb}
\usepackage{color}
\usepackage{framed}

% 定义pandoc代码高亮环境
\definecolor{shadecolor}{RGB}{248,248,248}
\newenvironment{Shaded}{\begin{snugshade}}{\end{snugshade}}
\DefineVerbatimEnvironment{Highlighting}{Verbatim}{commandchars=\\\{\}}
\newcommand{\KeywordTok}[1]{\textcolor[rgb]{0.13,0.29,0.53}{\textbf{#1}}}
\newcommand{\DataTypeTok}[1]{\textcolor[rgb]{0.13,0.29,0.53}{#1}}
\newcommand{\DecValTok}[1]{\textcolor[rgb]{0.00,0.00,0.81}{#1}}
\newcommand{\BaseNTok}[1]{\textcolor[rgb]{0.00,0.00,0.81}{#1}}
\newcommand{\FloatTok}[1]{\textcolor[rgb]{0.00,0.00,0.81}{#1}}
\newcommand{\ConstantTok}[1]{\textcolor[rgb]{0.00,0.00,0.00}{#1}}
\newcommand{\CharTok}[1]{\textcolor[rgb]{0.31,0.60,0.02}{#1}}
\newcommand{\SpecialCharTok}[1]{\textcolor[rgb]{0.00,0.00,0.00}{#1}}
\newcommand{\StringTok}[1]{\textcolor[rgb]{0.31,0.60,0.02}{#1}}
\newcommand{\VerbatimStringTok}[1]{\textcolor[rgb]{0.31,0.60,0.02}{#1}}
\newcommand{\SpecialStringTok}[1]{\textcolor[rgb]{0.31,0.60,0.02}{#1}}
\newcommand{\ImportTok}[1]{#1}
\newcommand{\CommentTok}[1]{\textcolor[rgb]{0.56,0.35,0.01}{\textit{#1}}}
\newcommand{\DocumentationTok}[1]{\textcolor[rgb]{0.56,0.35,0.01}{\textit{#1}}}
\newcommand{\AnnotationTok}[1]{\textcolor[rgb]{0.56,0.35,0.01}{\textbf{\textit{#1}}}}
\newcommand{\CommentVarTok}[1]{\textcolor[rgb]{0.56,0.35,0.01}{\textbf{\textit{#1}}}}
\newcommand{\OtherTok}[1]{\textcolor[rgb]{0.56,0.35,0.01}{#1}}
\newcommand{\FunctionTok}[1]{\textcolor[rgb]{0.00,0.00,0.00}{#1}}
\newcommand{\VariableTok}[1]{\textcolor[rgb]{0.00,0.00,0.00}{#1}}
\newcommand{\ControlFlowTok}[1]{\textcolor[rgb]{0.13,0.29,0.53}{\textbf{#1}}}
\newcommand{\OperatorTok}[1]{\textcolor[rgb]{0.81,0.36,0.00}{\textbf{#1}}}
\newcommand{\BuiltInTok}[1]{#1}
\newcommand{\ExtensionTok}[1]{#1}
\newcommand{\PreprocessorTok}[1]{\textcolor[rgb]{0.56,0.35,0.01}{\textit{#1}}}
\newcommand{\AttributeTok}[1]{\textcolor[rgb]{0.77,0.63,0.00}{#1}}
\newcommand{\RegionMarkerTok}[1]{#1}
\newcommand{\InformationTok}[1]{\textcolor[rgb]{0.56,0.35,0.01}{\textbf{\textit{#1}}}}
\newcommand{\WarningTok}[1]{\textcolor[rgb]{0.56,0.35,0.01}{\textbf{\textit{#1}}}}
\newcommand{\AlertTok}[1]{\textcolor[rgb]{0.94,0.16,0.16}{#1}}
\newcommand{\ErrorTok}[1]{\textcolor[rgb]{0.64,0.00,0.00}{\textbf{#1}}}
\newcommand{\NormalTok}[1]{#1}

% 中文支持
\usepackage{xeCJK}
\setCJKmainfont{SimSun} % 宋体
\setCJKsansfont{SimHei} % 黑体
\setCJKmonofont{FangSong} % 仿宋

% 页面设置
\geometry{
  a4paper,
  top=2cm,
  bottom=1cm,
  left=4cm,
  right=2cm,
  headheight=14pt
}

% 字体设置
\setmainfont{Times New Roman}

% 段落间距
\setlength{\parskip}{8pt}
\setlength{\parindent}{0pt}

% 行距设置
\onehalfspacing

% 页眉页脚设置 - 强化版
\pagestyle{fancy}
\fancyhf{} % 清除所有现有的页眉页脚
\fancyhead[C]{\fontsize{12pt}{14pt}\selectfont\rmfamily\thepage} % 页码居中显示在页眉
\renewcommand{\headrulewidth}{0pt} % 移除页眉横线
\renewcommand{\footrulewidth}{0pt} % 移除页脚横线

% 对第一页使用相同的页眉页脚样式
\fancypagestyle{plain}{%
  \fancyhf{}
  \fancyhead[C]{\fontsize{12pt}{14pt}\selectfont\rmfamily\thepage}
  \renewcommand{\headrulewidth}{0pt}
  \renewcommand{\footrulewidth}{0pt}
}

% 脚注设置
\renewcommand{\footnoterule}{\rule{5cm}{0.4pt}\vspace{0.5em}}
\setlength{\footnotesep}{1.5\baselineskip}
\renewcommand{\footnotesize}{\fontsize{10pt}{12pt}\selectfont}

% 其他设置
\setlength{\emergencystretch}{3em}
\providecommand{\tightlist}{\setlength{\itemsep}{0pt}\setlength{\parskip}{0pt}}


\begin{document}

% 强制启用fancy页面样式
\thispagestyle{fancy}




P95 - P113

\begin{center}\rule{0.5\linewidth}{0.5pt}\end{center}

\subsubsection{定义}\label{ux5b9aux4e49}

数列\(S_n\)满足\(\forall n \in \mathbb{N} \quad S_n := \sum_{j=1}^n a_j\),被称为(无穷)级数

\(\sum_{j=1}^\infty a_j:= \sum_{j \in \mathbb{N}}a_j := (S_n)_{n \in \mathbb{N}}\)

\begin{itemize}
\tightlist
\item
  \(a_j\)项
\item
  \((S_n)_{n \in \mathbb{N}}\) :部分和
\item
  \(\sum_{j  = k}^\infty a_j\):\((S_n)_{n \in \mathbb{N}}\)的余项
\end{itemize}

\textbf{级数是个特殊的数列}

\paragraph{级数收敛}\label{ux7ea7ux6570ux6536ux655b}

\begin{itemize}
\tightlist
\item
  级数收敛当且仅当:\(lim_{n->\infty}S_n = s \in K\),可以记作:\(\sum_{j = 1}^\infty a_j = s\)
\item
  同理,如果\(lim_{n->\infty}S_n = \pm\infty \in K\)
\end{itemize}

\paragraph{几何级数(Geometrische
Reihe)}\label{ux51e0ux4f55ux7ea7ux6570geometrische-reihe}

\(\forall |q| < 1\quad\sum_{j = 0}^\infty q^j = lim s_n = \frac{1}{1-q}\),标准几何级数收敛于\(\frac{1}{1-q}\)

\paragraph{调和级数(Harmonische
Reihe)}\label{ux8c03ux548cux7ea7ux6570harmonische-reihe}

\(\sum_{k=1}^\infty \frac{1}{k} = \infty\),调和级数是发散的

\paragraph{交错级数(Alternierend)}\label{ux4ea4ux9519ux7ea7ux6570alternierend}

\(\sum_{j=1}^{\infty} a_j \text{ 其中 } a_j \in \mathbb{R} \text{ 称为交错级数当且仅当} \forall j \in \mathbb{N} , \text{sgn}(a_{j+1}) = -\text{sgn}(a_j) \neq 0.\)

\textbf{莱布尼茨判别法}:如果交错级数
\(\sum_{n=1}^{\infty} (-1)^{n-1}a_n\) 满足以下条件:

\begin{enumerate}
\def\labelenumi{\arabic{enumi}.}
\tightlist
\item
  对所有 \(n\),有 \(a_n > 0\)
\item
  数列 \({a_n}\) 单调递减,即 \(a_{n+1} \leq a_n\)
\item
  \(\lim_{n \to \infty} a_n = 0\)
\end{enumerate}

那么,该交错级数\textbf{收敛}。

\subsubsection{级数运算法则}\label{ux7ea7ux6570ux8fd0ux7b97ux6cd5ux5219}

\paragraph{线性性}\label{ux7ebfux6027ux6027}

\(\forall \lambda,\mu \in \mathbb{C} \sum_{j = 1}^\infty\quad (\lambda a_j +\mu b_j) = \lambda\sum_{j = 1}^\infty a_j + \mu\sum_{j=1}^\infty b_j\)

\paragraph{单调性}\label{ux5355ux8c03ux6027}

\(\forall j \in \mathbb{N}, a_j, b_j \in \mathbb{R} \land a_j \leq b_j \Rightarrow \sum_{j=1}^{\infty} a_j \leq \sum_{j=1}^{\infty} b_j.\)

\paragraph{余项趋于0}\label{ux4f59ux9879ux8d8bux4e8e0}

\(\lim_{n \to \infty} a_n = 0, \forall n \in \mathbb{N} \sum_{j=n+1}^{\infty} a_j \text{ konvergiert.}\)\\
\(\text{Mit } S := \sum_{j=1}^{\infty} a_j, r_n := \sum_{j=n+1}^{\infty} a_j \text{ gilt:}\)

\(\forall n \in \mathbb{N} , S = s_n + r_n\)

\(\lim_{n \to \infty} r_n = 0.\)

\subsubsection{收敛性判据}\label{ux6536ux655bux6027ux5224ux636e}

\begin{itemize}
\item
  \textbf{有界判别法:部分和有界,则级数收敛}

  \begin{itemize}
  \tightlist
  \item
    \(a_j \in \mathbb{R}^+_0, s_n := \sum_{j=1}^{n} a_j\)\\
  \item
    \(\lim_{n \to \infty} s_n = \sup\{s_n : n \in \mathbb{N}\}\) ,收敛
  \item
    \(\lim_{n \to \infty} s_n = \infty\),发散
  \end{itemize}
\item
  \textbf{比较判别法:}
\end{itemize}

\(\exists k \in \mathbb{N} , \forall j \geq k , |a_j| \leq |b_j|, (a_j, b_j \in \mathbb{C}).\)

\((a) \sum_{j=1}^{\infty} |b_j| \text{ 收敛} \Rightarrow \sum_{j=1}^{\infty} a_j \text{ 收敛,且} \left|\sum_{j=k}^{\infty} a_j\right| \leq \sum_{j=k}^{\infty} |b_j|. \quad (7.78)\)

\((b) \sum a_j \text{ 发散} \Rightarrow \sum |b_j| \text{ 发散}.\)

\begin{itemize}
\tightlist
\item
  \textbf{莱布尼茨判别法:}
\end{itemize}

\(\sum_{j=1}^{\infty} a_j \text{是交错级数 }(|a_n|)_{n \in \mathbb{N}} \text{单调递减且} \lim_{n \to \infty} a_n = 0 \text{则} \sum_{j=1}^{\infty} a_j \text{收敛且}\):

\(\forall n \in \mathbb{N} , \exists_{0 < \theta_n < 1} , r_n := \sum_{j=n+1}^{\infty} a_j = \theta_n a_{n+1}\)

对于任意自然数n,存在一个介于0和1之间的数θₙ,使得第n项之后的所有项之和rₙ等于θₙ乘以aₙ₊₁。
这个结论提供了交错级数余项的估计,表明余项的绝对值不会超过下一项的绝对值,这对于计算误差很有帮助。

\begin{itemize}
\tightlist
\item
  \textbf{根值判别法(Wurzelkrit)}
\end{itemize}

如果对于级数 \(\sum_{j=1}^{\infty} a_j\),极限
\(\lim_{j \to \infty} \sqrt[j]{|a_j|} = L\) 存在,那么:

\begin{itemize}
\tightlist
\item
  如果 \(L < 1\),则级数绝对收敛
\item
  如果 \(L > 1\),则级数发散
\item
  如果 \(L = 1\),则判别法不能确定收敛性
\end{itemize}

\(\exists 0<q<1 \left(\sqrt[n]{|a_n|} \leq q < 1 \text{ 对几乎所有 } n \in \mathbb{N}\right)\)

\(\Rightarrow \sum_{j=1}^{\infty} a_j \text{ 绝对收敛}\)

\(\{n \in \mathbb{N} : \sqrt[n]{|a_n|} \geq 1\} = \infty \Rightarrow \sum_{j=1}^{\infty} a_j\)

\begin{itemize}
\tightlist
\item
  \textbf{比值判别法(Quot.krit)}:
\end{itemize}

\subsubsection{绝对收敛(Abs.
Konv.)}\label{ux7eddux5bf9ux6536ux655babs.-konv.}

\begin{itemize}
\tightlist
\item
  \textbf{定义}:\(\sum_{j=1}^{\infty} a_j \text{ 称为绝对收敛} :\Leftrightarrow \sum_{j=1}^{\infty} |a_j| \text{ 收敛}\)
\end{itemize}

\(\sum_{j=1}^{\infty} a_j\) 绝对收敛 \(\Rightarrow\)
\(\sum_{j=1}^{\infty} a_j\) 收敛,且满足三角不等式,即

\(\sum_{j=1}^{\infty} c_j \text{ 收敛} \wedge \forall j \in \mathbb{N}, |a_j| \leq c_j \Rightarrow \sum_{j=1}^{\infty} a_j \text{ 绝对收敛}\)

这表述了比较判别法的一个版本:如果有一个收敛的正项级数
\(\sum_{j=1}^{\infty} c_j\),并且对于所有
\(j \in \mathbb{N}\),\(|a_j| \leq c_j\),那么级数
\(\sum_{j=1}^{\infty} a_j\) 绝对收敛。

\subsubsection{b进制表示法的实数}\label{bux8fdbux5236ux8868ux793aux6cd5ux7684ux5b9eux6570}

解释:\(\sum_{n=0}^{\infty} 2^{-(2n+1)} = \frac{2}{3}\)(练习题)。

这个级数是\(\sum_{n=0}^{\infty} 2^{-(2n+1)}\),可以理解为:\\
\(\frac{1}{2^1} + \frac{1}{2^3} + \frac{1}{2^5} + \frac{1}{2^7} + ...\)

\textbf{这是一个几何级数的变形。通过数学推导(可作为练习题),这个无穷级数的和等于\(\frac{2}{3}\)。}

每个自然数n∈N恰好有2种十进制(即10进制)表示法。例如:\\
2 = 2.0 = 1.9 = 1 + \(\sum_{n=1}^{\infty} 9 \cdot 10^{-n}\) (7.71)\\
= 1 + 9 · \(\left(\frac{1}{1 - \frac{1}{10}} - 1\right)\)\\
= 1 + 9 · \(\frac{1}{9}\). (7.95)

\subsection{逐点收敛与一致收敛}\label{ux9010ux70b9ux6536ux655bux4e0eux4e00ux81f4ux6536ux655b}

\begin{itemize}
\item
  \textbf{背景举例:}

  \begin{itemize}
  \tightlist
  \item
    考虑在定义域\([0,1]\)上的函数序列\((f_n)\),其中每个函数\(f_n:[0,1] \to \mathbb{R}\)的定义为:\(f_n(x) = x^n\)
  \item
    n趋近于无穷时候,观察随着n的增加,函数\(f_n(x)\)
    在整个定义域\([0,1]\)内,逐渐趋近于0,表达为:
    \(0 \leq x \leq 1 \quad lim_{n \to \infty}f_n(x) = lim_{n \to \infty}x^n = 0\)
  \item
    \textbf{因此,极限函数 \(f\) 可以定义 为:} \[f(x) = \begin{cases} 
      0 & \text{if } 0 \leq x < 1 \\
      1 & \text{if } x = 1
      \end{cases}\]
  \item
    \textbf{简单解释}:逐点收敛意味着在定义域内的每一个固定点上,随着~n增大,\(f_n\)\hspace{0pt}~的值逐渐接近~\(f\)~的值。但这种接近可能\textbf{在不同点上的速度非常不同}
  \end{itemize}
\item
  \textbf{定义:}

  \begin{itemize}
  \tightlist
  \item
    \textbf{逐点收敛:} 函数序列\((f_n)\)逐点收敛于\$
    \(f: M \to \mathbb{K}\) 当且仅当
    \(lim_{n \to \infty}f_n(z) = f(z)\),也可以写为当且仅当: \[
      \forall z \in M\quad\forall \epsilon \in \mathbb{R}^+\quad\exists N \in \mathbb{N}\quad\forall n > N\quad|f_n(z) - f(z)| < \epsilon
      \]
  \item
    \textbf{一致收敛:}
    函数序列\((f_n)\)一致(\textbf{gleichmäßig})收敛于\(f: M \to \mathbb{K}\)
    当且仅当: \[
      \forall \epsilon \in \mathbb{R}^+\quad\exists N \in \mathbb{N}\quad\forall n > N\quad\forall z \in M\quad|f_n(z) - f(z)| < \epsilon
      \]
  \item
    \textbf{简单解释}:一致收敛意味着整个定义域内所有点上,函数序列~\(f_n\)\hspace{0pt}~同时且以相同的速度向~\(f\)~接近。换句话说,在足够大的~n下,无论选择定义域中的哪个~x,\(f_n(x)\)~都会非常接近~\(f(x)\)
  \end{itemize}
\end{itemize}

\subsection{幂级数(Potenzreihen)}\label{ux5e42ux7ea7ux6570potenzreihen}

\begin{itemize}
\item
  \textbf{定义:}

  \begin{itemize}
  \item
    \textbf{函数级数:}令
    \(\emptyset \neq M \subseteq \mathbb{C}\),函数级数,\((f_n): M \to \mathbb{K}\)
    : \(\sum_{j=1}^\infty f_j := (S_n)_{n \in \mathbb{N}}\)
  \item
    \textbf{幂级数:}
    \(f_n:\mathbb{K} \to \mathbb{K}\quad f_n(z) = a_n \in \mathbb{K}\quad\sum_{j=0}^\infty a_jz^j := \sum_{j = 0}^\infty f_j\)

    \begin{itemize}
    \item
      j 从 0开始,意味着幂级数会有常数项**
    \item
      \(f_j(z) = a_jz^j\),处理幂级数时候,必须从上下文中判定是数字还是函数
    \end{itemize}
  \end{itemize}
\item
  \textbf{幂级数展开:} 当且仅当 \((S_n)\) \textbf{逐点收敛(p.w. gegen)}
  于 \(f:M \to \mathbb{K}\),函数 \(f\)
  可以表示成为无穷级数的形式,意味着级数是函数 \(f\) 的幂级数展开。
\item
  \textbf{一致收敛准则(充分条件)}:如果有\(a_j \in \mathbb{R}^+\quad\sum_{j=1}^\infty a_j\)这个正项级数收敛,\(\forall z \in M\quad\forall j \in \mathbb{N}\quad|f_j(z)| \leq a_j\)
  ---- \textbf{Weierstrass判别法或者M判别法} 函数序列
  \((f_j)\)被一个收敛的正项级数 \((a_j)\)所控制
\item
  \textbf{一致收敛的连续保持性:} 如果函数\(f_j\)在
  \(\zeta(\zeta \in M)\) 处连续(stetig),并且\(\sum_{j = 1}^\infty f_j\)
  一致收敛于 \(f\),那么 \(f\) 在 zeta 处也保持连续
\item
  \textbf{Python示例:}
\end{itemize}

\begin{Shaded}
\begin{Highlighting}[]
\ImportTok{import}\NormalTok{ numpy }\ImportTok{as}\NormalTok{ np}
\ImportTok{import}\NormalTok{ matplotlib.pyplot }\ImportTok{as}\NormalTok{ plt}

\CommentTok{\# 计算幂级数 e\^{}x = 1 + x + x\^{}2/2! + x\^{}3/3! + ...}
\KeywordTok{def}\NormalTok{ exp\_series(x, terms}\OperatorTok{=}\DecValTok{10}\NormalTok{):}
\NormalTok{    result }\OperatorTok{=}\NormalTok{ np.zeros\_like(x, dtype}\OperatorTok{=}\BuiltInTok{float}\NormalTok{)}
    \ControlFlowTok{for}\NormalTok{ n }\KeywordTok{in} \BuiltInTok{range}\NormalTok{(terms):}
\NormalTok{        result }\OperatorTok{+=}\NormalTok{ x}\OperatorTok{**}\NormalTok{n }\OperatorTok{/}\NormalTok{ math.factorial(n)  }\CommentTok{\# 添加每一项}
    \ControlFlowTok{return}\NormalTok{ result}

\CommentTok{\# 使用示例}
\NormalTok{x }\OperatorTok{=}\NormalTok{ np.linspace(}\OperatorTok{{-}}\DecValTok{2}\NormalTok{, }\DecValTok{2}\NormalTok{, }\DecValTok{100}\NormalTok{)}
\NormalTok{plt.plot(x, np.exp(x), }\StringTok{\textquotesingle{}b{-}\textquotesingle{}}\NormalTok{, label}\OperatorTok{=}\StringTok{\textquotesingle{}真实函数 e\^{}x\textquotesingle{}}\NormalTok{)}
\NormalTok{plt.plot(x, exp\_series(x, }\DecValTok{5}\NormalTok{), }\StringTok{\textquotesingle{}r{-}{-}\textquotesingle{}}\NormalTok{, label}\OperatorTok{=}\StringTok{\textquotesingle{}5项近似\textquotesingle{}}\NormalTok{)}
\NormalTok{plt.legend()}
\NormalTok{plt.title(}\StringTok{\textquotesingle{}e\^{}x的幂级数展开\textquotesingle{}}\NormalTok{)}
\NormalTok{plt.show()}
\end{Highlighting}
\end{Shaded}

\begin{itemize}
\item
  \textbf{收敛半径:}

  \begin{itemize}
  \item
    \textbf{定义:}对于每一个幂级数\(\sum_{j = 0}^\infty a_jz^j\),存在一个\(r \in [0,+\infty)\),称为该级数的收敛半径(KR),使得当\(z \in \mathbb{K}\)时候:

    \begin{itemize}
    \item
      \(|z| < r\):幂级数收敛
    \item
      \(|z| > r\):幂级数发散
    \item
      \(|z| = r\):幂级数收敛性需要逐点检查,可能某些点收敛,某些点发散
    \end{itemize}
  \item
    \textbf{判定方法:}

    \begin{itemize}
    \item
      \textbf{比值法}:\(\rho = lim_{j \to \infty} |\frac{a_{j+1}}{a_j}|\)
      有极限,收敛半径等于 \(\frac{1}{\rho}\)
    \item
      \textbf{根值法}:\(L = lim_{j \to \infty}\sqrt[n]{a_n}\)
      ,收敛半径等于 \(\frac{1}{L}\)
    \end{itemize}
  \end{itemize}

  \textbf{例题:Bsp.
  8.11,用根值法计算出收敛半径后,针对常数项的一些特殊取值,如0,1,-1等,转换为常见的调和级数,交错级数,p级数等判断敛散性,再把区间进行汇总给出什么条件下,什么区间收敛}
\end{itemize}

\subsection{指数函数(Exponentialfunktionen)}\label{ux6307ux6570ux51fdux6570exponentialfunktionen}

\begin{itemize}
\item
  \textbf{定义}:

  \begin{itemize}
  \item
    \(e = lim_{n \to \infty} (1 + \frac{1}{n})^n\)
  \item
    \(exp: \mathbb{C} \to \mathbb{C}, exp(z) := \sum_{n = 0}^\infty \frac{z^n}{n!} = 1+ z + \frac{z^2}{2!}\)
  \end{itemize}
\item
  \textbf{常用极限}

  \begin{itemize}
  \item
    \textbf{洛必达法则可以算}:

    \begin{itemize}
    \item
      \(\lim_{z\to 0} \frac{e^z-1}{z} = 1, \quad (z \in M := \mathbb{C}\setminus\{0\})\)
    \item
      \(\lim_{x\to 0} \frac{\ln(1+x)}{x} = 1, \quad (x \in M := [-1,\infty)\setminus\{0\})\)
    \item
      \(\lim_{x\to 0} (1+\xi x)^{\frac{1}{x}} = e^\xi, \quad (x \in M := \{x \in \mathbb{R}: 1+\xi x > 0\}\setminus\{0\})\)
    \end{itemize}
  \item
    \textbf{需要记忆:}

    \begin{itemize}
    \item
      \(\lim_{x\to 0} \ln(1+\xi x)^{\frac{1}{x}} = \xi, \quad (x \in M := \{x \in \mathbb{R}: 1+\xi x > 0\}\setminus\{0\})\)
    \item
      \(\lim_{n\to\infty} (1+\frac{x}{n})^n = e^x = \sum_{n=0}^{\infty} \frac{x^n}{n!}\)
    \end{itemize}
  \end{itemize}
\end{itemize}

\end{document}